\thispagestyle{empty}
\vspace*{1.0cm}

\begin{center}
    \textbf{Abstract} \label{abstract}
\end{center}

\vspace*{0.5cm}

Machine learning (ML) models are notoriously resource-intensive.
Given their widespread application across every-day edge devices,
the need to reduce their memory and computing requirements is becoming ever more pressing.

Despite the said resource-intensiveness of ML models, at the same time 
they offer the main ingredient for the remedy to the malady - redundancy - 
which can be exploited to reduce their memory usage.

While the redundancy exploitation of ML models is already a common 
technique that comes in different forms, starting from pruning and
ending with ..., quantization presents itself as an especially promising 
area especically in the sense of learned quantization - 
the process of making ML models learn their optimal quantization parameters
on their own.

Hence, the current work employs two techniques that bypass the main issue of learned quantization,
that is, the non-differentiability of rounding operations.
While the first technique involves custom loss functions that directly take into account
quantization goals, the second approach incorporates a custom gradient calculation that is
easily integratable to various training scenarios.

As a result of these two techniques, a memory usage reduction of up to ..x is obtained 
on MNIST, CIFAR10, and Imagenette. 

\begin{enumerate}
    \item The problem:
        \begin{enumerate}
            \item High memory usage
            \item Models are redundant
        \end{enumerate}

    \item Why it's an interesting problem
        \begin{enumerate}
            \item \{I need something about current techniques not being enough\}
            \item \{Also something about learned quantization specifically that's not enough\}
        \end{enumerate}
        Despite the said resource-intensiveness of ML models, they 

    \item What our solution achieves
        \begin{enumerate}
            \item Custom loss functions that encourage quantization
            \item Custom gradient calculation that encourages quantization only where it's necessary 
        \end{enumerate}            
    \item What follows from our solution
        \begin{enumerate}
            \item Reduced memory requirement of X with accuracy degradation within Y \%.
        \end{enumerate}   
\end{enumerate}
